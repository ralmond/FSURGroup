% This is a "bare-bones" thesis template file.  For examples of how to
% use a few more LaTeX features, look in the 'sample' folder.  Read
% the User Guide for documentation of the 'fsuthesis' class features.
% Follow the FSU 'Guidelines and Requirements for Electronic Theses,
% Treatises, and Dissertations (ETDs)' document for all entries.

\documentclass[11pt,expanded,copyright]{fsuthesis}

% Additional packages may be loaded here.

% Do NOT use 'geometry', 'setspace', or 'tocloft' packages! They will
% mess up the spacing provided by 'fsuthesis'.

% If using the BibLaTeX package for generating a bibliography or
% references section, uncomment the following 5 lines. This is a
% basic starting point, and you may want other package options.
% You'll also need to uncomment the '\printbibliography' line
% toward the end of this file. Refer to the biblatex package
% documentation for information. The file 'myrefs.bib' contains
% a few sample references. 

%\usepackage[american]{babel}
%\addto\captionsamerican{\renewcommand*{\contentsname}{Table of Contents}}
%\usepackage{csquotes}
%\usepackage{biblatex}
%\addbibresource{myrefs.bib}

%
% Update these entries. See the User Guide for more information.
%
\title{This Is My Title:\protect\\And This Is Its Second Line}
\author{Viktor Spoyles}
\college{College of Extraterrestrial Recreation}
\department{Department of Human Slacking}  % Delete if no department
\manuscripttype{Dissertation}              % [Thesis, Dissertation, Treatise]
\degree{Master of My Domain}               % [Master of Science, Doctor of Philosophy...]  
\degreeyear{2021}
\defensedate{October 31, 2021}

%
% If creating a PDF, you may want to uncomment these and enter
% appropriate metadata for your document. See the User Guide.
%
%\subject{My Topic}
%\keywords{keyterm1; keyterm2; keyterm3; ...}
%

%
% Update to use the real names and positions of each person.
% See FSU's 'Guidelines and Requirements' document for the proper
% formats and titles.
%
\committeeperson{Faux Causson Yorverk}{Professor Directing Traffic}
\committeeperson{Verda Boizaar}{University Representative}
\committeeperson{Beauxeau D'Claune}{Committee Member}
\committeeperson{Arlip Zarseeld}{Committee Member}

\clubpenalty=9999
\widowpenalty=9999

\begin{document}

\frontmatter
\maketitle
\makecommitteepage

%\begin{dedication}
%\end{dedication}

%\begin{acknowledgments}
%\end{acknowledgments}

\tableofcontents
%\listoftables
%\listoffigures
%\listofmusex

%\begin{listofsymbols}
%\end{listofsymbols}

%\begin{listofabbrevs}
%\end{listofabbrevs}

\begin{abstract}
This is my abstract.
\end{abstract}

\mainmatter

\input chapter1
%\input chapter2
%\input chapter3

%\appendix
%\input appendix1

% You have your choice of bibliography sections: either
% hand-crafted or BibLaTeX. The hand-crafted References
% section is enabled by default. Remove these entries if
% you're using BibLaTeX.

% This is the "hand-crafted" bibliography/references section:
\begin{references}
Mybib, Sample. \textit{An Example of a Bibliographic Entry
 Created Manually}. Tallahassee, Florida: Fornish and Frak, 2010.

Smith, Marigold. \textit{Lots and Lots of Bibliographic Entries
 and How to Display Them}. Tallahassee, Florida: Gibson and Goulash, 2010.
\end{references}

% Or use the BibLaTeX bibliography package, uncomment the lines in the
% preamble, and uncomment the \printbibliography line below. View the
% file 'myrefs.bib' to get a feel for what these entries may look
% like. See the document in the 'sample' folder for more citation and
% BibLaTeX examples.

%\printbibliography

\begin{biosketch}
This is my biography.
\end{biosketch}

\end{document}
