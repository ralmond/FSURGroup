%%% For this chapter, I need a few extra definitions.  First, I create a
%%% new macro so that I can easily type the BibTeX and BibLaTeX logos.

\def\BibTeX{{\rmfamily B\kern-.05em%
   \textsc{i\kern-.025em b}\kern-.08em\TeX}}
\def\BibLaTeX{{\rmfamily B\kern-.05em%
   \textsc{i\kern-.025em b}\kern-.08em\LaTeX}}

%%% Next I create a new environment called 'narrowref' simulating the
%%% appearance of the 'references' environment.  The margins are
%%% narrowed to set off the material a little better.  I also invoke
%%% the 'singlespaced' environment to demonstrate the look of the
%%% actual output of the references in case I'm processing this
%%% document using 'expanded' mode.  If you'd like to create your own
%%% environments for special purposes, refer to the references for the
%%% details.  For aspiring LaTeX experts, note that I reset
%%% the \parskip parameter *after* beginning the first paragraph
%%% with \leavevmode.  This is because I only want to use the new
%%% paragraph spacing between paragraphs contained *within* this
%%% environment, and the default paragraph spacing outside the
%%% environment.  (The \parskip glue is added before the start of a
%%% new paragraph, not at the end of the old paragraph.)

\newenvironment{narrowref}{
  \begin{singlespaced}
  \frenchspacing
  \setlength{\leftskip}{4em}
  \setlength{\rightskip}{3em}
  \setlength{\parindent}{-2em}
  \addvspace{0.5\baselineskip}
  \leavevmode
  \setlength{\parskip}{0.6\baselineskip}\ignorespaces}
 {\end{singlespaced}\addvspace{.5\baselineskip}}

%%% Now let the chapter begin.

\chapter{Citations and References}
In this chapter I demonstrate how to use features of \LaTeX{} and
the \pkg{fsuthesis} class to help me create a list of references or
bibliography.  There are three options: (1)~use the \lit{references}
environment and enter bibliographic entries with styling that I create
myself, or (2)~use the latest bibliography system called \BibLaTeX{}
to manage a database of references, using \verb|\cite| macros within
your text to create citations, or (3)~use the older \BibTeX{} system
to order and style the bibliographic entries according to a
pre-defined style sheet. Option~(2) is preferable to (3), unless you
need to use an older style macro package that hasn't been converted
to \BibLaTeX\ yet.

\section{Using the \texttt{references} Environment}
Let's start simply.  Assuming we don't have many citations, we'll
create a references section manually using the
\verb|\begin{references}| environment provided by the
\textsf{fsuthesis} class.  If your discipline has a style
guide for presenting references, you should use that information to
create your own entries.  Here are some example entries with
explicit formatting specified:
\begin{verbatim}
\begin{references}
Picaut, J., F. Masia, and Y. du Penhoat, 1997: An advective-reflective
conceptual model for the oscillatory nature of the ENSO.
\textit{Science}, \textbf{277}, 663--666.

Yasunari, T., 1990: Impact of Indian monsoon on the coupled
atmosphere/ocean system in the tropical Pacific.
\textit{Meteor. Atmos. Phys.}, \textbf{44}, 19--41.
\end{references}
\end{verbatim}

Note that I had to specify italics and bold-facing myself, according
to the style guide that I'm using.  (If you don't have a
discipline-specific style guide, see \cite{Anonymous:1993:CMS}
or \cite{Anonymous:2009:PMA} for lots of bibliographic examples.)
The \lit{references} environment provides exdented entries, spacing,
and a heading, but the rest of the formatting is up to you.  Likewise,
citations of these references within your document must be formatted
manually.  Once processed, the reference entries above look like the
following:
\begin{narrowref}
Picaut, J., F. Masia, and Y. du Penhoat, 1997: An advective-reflective
conceptual model for the oscillatory nature of the ENSO.
\textit{Science}, \textbf{277}, 663--666.

Yasunari, T., 1990: Impact of Indian monsoon on the coupled
atmosphere/ocean system in the tropical Pacific.
\textit{Meteor. Atmos. Phys.}, \textbf{44}, 19--41.
\end{narrowref}

%%% My \BibTeX macro fails me in this section title, because there are
%%% no smallcaps bold-faced fonts provided by default with my LaTeX
%%% installation.  For this one place, I do something slightly different.

\section{Citations and B{\small IB}\LaTeX}
If your thesis or dissertation does not have many citations or
references, then the \lit{references} environment may be all you
need. However, if you have more than a handful of citations, you owe
it to yourself to invest a little more energy into learning about the
powerful, time-saving features of \BibLaTeX.  To use \BibLaTeX,
bibliographic entries are added to an external file with attributes
identifying elements of the entry, such as authors, titles, journals,
etc.  You may then cite a reference in your document using its unique
key.  Many disciplines have developed large \BibTeX/\BibLaTeX{}
databases already, so if you're lucky, you only need download a
pre-built file ready to go.  You can always add a few more references
if those entries don't already exist in the file you
download. (\BibTeX{} databases are compatible with \BibLaTeX{}, so
very few changes are necessary to use the newer system. However,
\BibLaTeX{} provides many more features and options, so not all
\BibLaTeX{} bibliography entries will be backwards compatible with
\BibTeX).

For this sample document, I have created a small \BibLaTeX{}
bibliography database in a file called \lit{myrefs.bib} which is
excerpted from a larger collection of pre-generated \TeX-related
entries I downloaded from the web.  If your discipline does not
already distribute \BibLaTeX{} databases publicly, it's just a bit more
typing to create your own \BibLaTeX{} file.  As an example, here's
what a \BibLaTeX{} entry might look like:
\begin{verbatim}
@Article{Picaut:1997,                  
  author = "J. Picaut and F. Masia and Y. du Penhoat",
  title = "An advective-reflective conceptual model for
    the oscillatory nature of the ENSO",
  volume = "277",
  year = "1997",
  journal = "Science",
  pages = "663--666"
}
\end{verbatim}

Another advantage of the \BibLaTeX{} approach is that you can continue
to add to this database throughout your professional career, creating
entries as you read books and journal articles you may want to
reference in the future.  I encourage you to refer to the
standard \LaTeX{} references (e.g., \cite{Lamport:1994:LDP}
and \cite{Kopka:2004:GLT}) or to search the web for further
information on creating your own \BibLaTeX{} database.

Once you've created or downloaded a \BibLaTeX{} database, you may cite a
document using its unique \textit{key} as an argument to
the \LaTeX{} \verb|\cite| macro.  For example, in the previous
paragraph, I used the commands \verb|\cite{Lamport:1994:LDP}| and
\verb|\cite{Kopka:2004:GLT}|, where \lit{Lamport:1994:LDP} and
\lit{Kopka:2004:GLT} are the keys for their respective documents.
(The keys are defined in the \BibLaTeX\ database file.)  By making these
citations, the bibliographic entries for these documents will be
pulled from my \BibLaTeX{} database and added to the bibliography
automatically.

I can also add entries to the bibliography without having citations in
my text by using the \verb|\nocite{key}| command with the
desired \textit{key}.  If I want every entry in my \BibLaTeX{} file
inserted into my bibliography, then I can issue the
command \verb|\nocite{*}| as a shortcut.  (You don't have to worry
about citations being redundant: bibliographic entries will be
included only once no matter how many times they may be \verb|\cite|d
or \verb|\nocite|d in your document.)

For a comparison of the \BibTeX{} and \BibLaTeX{} systems, the
links below provide useful information.
\begin{itemize}\small
\item \textsf{http://tex.stackexchange.com/questions/25701/bibtex-vs-biber-and-biblatex-vs-natbib}
\item \textsf{http://tex.stackexchange.com/questions/5091/what-to-do-to-switch-to-biblatex}
\end{itemize}

\section{Alternative Citation and Bibliography Formats}
The pre-formatted version of this document uses the default citation
and bibliography formatting supplied by \LaTeX\ and \BibLaTeX.  There
are a few options to alter the appearance of citations and the
bibliography.  (See one of the references for your choices.)  However,
many disciplines and journals have created their own
advanced \BibTeX{} or \BibLaTeX{} formatting styles, and you may want
to use one of these in your document.

\subsection{The \pkg{biblatex-chicago} Package}
While the older \BibTeX\ \pkg{natbib} package provides an optional
bibliography style that closely follows the so-called ``Chicago
style'' (see \cite{Anonymous:1993:CMS}),
the \BibLaTeX{} \pkg{biblatex-chicago} package provides many more
features and stricter stylistic adherence over a wider range of types
of references. Include the
line \verb|\usepackage[style=chicago]{biblatex}| in the document
preamble to invoke this style option.

Like most \BibLaTeX{} styles, this package makes use of the
``back-end'' processor called \lit{biber} rather than the
older \lit{bibtex} program. The major \TeX{} distributions already
include the required programs. However, you may need to configure your
environment to run \lit{biber} rather than \lit{bibtex} to process
your document.

\subsection{The \pkg{biblatex-apa} Package}
For compatibility with the seventh edition of APA bibliography and
citation styles, use the \pkg{biblatex-apa} package, accessed using
the \verb|\usepackage[style=apa]{biblatex}| command. For sixth edition
compatability, see the \pkg{apacite} package below.

\subsection{The \pkg{natbib} B{\small IB}\TeX{} Package}
The \pkg{natbib} package is an older \BibTeX-style macro package,
activated by including the line \verb|\usepackage{natbib}| in the
document preamble, which provides a common alternative to the
default \LaTeX\ bibliography and citation styles.  Among other things,
this package will allow you to make author--year citations in your
document automatically (e.g., ``Fargunkle et al., (2001)'') without
any changes to the \BibTeX{} database file.  In addition to the
default \LaTeX\ \verb|\cite| macro, the package provides several new
macros for creating citations in different contexts.

This document's driver file (\lit{thesis.tex}) has options for trying
out the \pkg{natbib} package.  Look for the \verb|\usepackage{natbib}|
line near the top of the file, and then scroll down to the
bibliography section for additional style selections you must specify
for the bibliography itself.  For a bibliography formatted according
to the style of the Association for Computing Machinery, for example,
you would issue the command \verb|\bibliographystyle{acm}| before
the \verb|\bibliography| command.  See \cite{Kopka:2004:GLT} (or
search the web) for more information on \pkg{natbib}.

\subsection{The \pkg{apacite} Package}
The APA (American Psychological Association) created another
commonly-used style guide for citations and bibliographic formatting.
The \LaTeX\ package \pkg{apacite} was designed to follow the APA
guidelines, sixth edition. (For seventh edition compatibility, you
should instead use the \BibLaTeX{} \lit{style=apa} option.)  The
package provides its own set of macros for specifying citations in the
document text, but it also has a mode for compatibility with the
popular \pkg{natbib} package.  With this compatibility mode enabled,
you may use \pkg{natbib}-style citation commands in your document to
generate the APA-style citations and bibliography.  In addition, using
this compatibility mode means that you may switch between these two
package alternatives without having to re-write the citations in your
document.

If you use the \pkg{apacite} package, and if you have also enabled
the \pkg{hyperref} package for your document, you \emph{must} enable
\lit{natbibapa} compatibility mode, or citations will give you
cryptic error messages when your document is processed.  Load the
package using the following line:
\begin{verbatim}
\usepackage[natbibapa]{apacite}
\end{verbatim}
The corresponding bibliography style would be set by the following 
command line:
\begin{verbatim}
\bibliographystyle{apacite}
\end{verbatim}


